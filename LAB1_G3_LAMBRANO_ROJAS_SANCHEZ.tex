\documentclass[journal]{IEEEtran}
\usepackage[spanish]{babel}
\usepackage[utf8]{inputenc}
\usepackage{graphicx}
\graphicspath{ {images/} }
%\usepackage{enumerate}
\usepackage{amsmath}
\usepackage{mathtools}
\usepackage{float}
\usepackage{amssymb}
\usepackage{listings}
%\usepackage{caption}
\usepackage[justification=centering]{caption}
\usepackage[font=scriptsize,labelfont=bf]{caption}
\renewcommand{\IEEEkeywordsname}{Palabras clave}
%\renewcommand{\table}{TABLA}

\bgroup
\def\arraystretch{1.5}


\lstset{
basicstyle=\footnotesize
%,
%numbers=left
}

\begin{document}

\title{\textbf{Lab1: Power in home appliances}}
\author{
  \IEEEauthorblockN{ Jorge Lambra\~no$^3$, Julian Rojas$^2$, 
  Juan Sánchez$^3$\vspace{0.2cm}}\\
  \IEEEauthorblockA{\texttt{\small{$^1$jelambrano, $^2$drojasj,
   $^3$paradac @uninorte.edu.co
   }}}
}

\markboth{Laboratorio 1: Power in home appliances}%
{Shell \MakeLowercase{\textit{et al.}}: Bare Demo of IEEEtran.cls 
for Journals}
	
\maketitle

\begin{abstract}
This report presents the design and implementation 
of a security box and a dimmer circuit using DIACs 
and TRIACs. The reader also can find  the validation 
review with the theoretical model seen in class.    
\end{abstract}

%-----------------------------------------------------------

\begin{IEEEkeywords}  
Current, Power, Power Factor, Voltage, Waveform.
\end{IEEEkeywords}

%-----------------------------------------------------------
\IEEEpeerreviewmaketitle
%--------------------------------------------------------------------------

\section{INTRODUCCIÓN}

El objetivo principal de esta práctica es realizar 
un análisis de la forma de onda y las mediciones 
de tres tipos diferentes de carga. Dentro de la caja 
de seguridad hay un fusible para proteger el equipo 
de cualquier corto circuito. Usamos una resistencia 
de 1 $\Omega$ y 10 W para medir la corriente dividiendo 
la tensión entre 1 para obtener el valor actual. 
Todos los dispositivos electrónicos están compuestos 
de resistencias, condensadores e inductancias, un 
soldador es una carga lineal resistiva, requieren calor 
para funcionar. El voltaje medido sería el mismo que 
el de la fuente, pero la corriente variará según el 
consumo de energía del dispositivo. Esperamos la misma 
forma de onda para el voltaje y la corriente y ningún 
cambio de fase entre ellos.\\

Un taladro es una carga lineal inductiva, basado 
en el hecho de que los motores están hechos de 
bobinas. Debería haber un cambio de fase entre el 
voltaje y la corriente. La computadora portátil 
es una carga no lineal, la forma de onda del voltaje 
sería la misma, pero esperamos una forma diferente 
para la corriente. \\

La segunda parte consiste en diseñar y desarrollar 
un controlador de CA compuesto por un DIAC y un 
QUADRAC. Este tipo de circuito puede cambiar el 
voltaje RMS en los terminales de una carga lineal 
manipulando el ángulo de disparo del QUADRAC usando 
un potenciómetro. La carga ya no será lineal debido 
al circuito resultante entre la bombilla y el 
controlador AC.


%--------------------------------------------------------------------

\section{PROCEDIMIENTO Y ANÁLISIS DE RESULTADOS}

\subsection{Cálculos de Potencia}

El objetivo de esta práctica es medir el consumo eléctrico 
de dispositivos cotidianos, tales como computadores 
portátiles, un pequeño taladro y un cautil. Teniendo en 
cuenta los valores altos de tensión y corriente que se 
pueden encontrar en instalaciones eléctricas domésticas, se
tomaron todas las precausiones necesarias para garantizar 
seguridad a los dispositivos, instrumentos de medición y 
a las personas que realizan la medición. Para poder 
obtener los valores de tensión, corriente y corriente se 
diseñó un circuito que junto con el osciloscopio, es capaz de
permite la obtención de valores de tensión y corriente. El 
diseño del circuito se encuentra ilustrado en la Figura 
\ref{circuit_diagram}.

%The purpose of this practice is to measure the 
%power consumption on electrical home devices, such 
%as a laptop, a drill and a soldering iron. Taking into 
%account that they work with high values of voltage 
%and current compared with previous labs, precautions 
%were taken in order to protect the devices and our 
%integrity. The Figure \ref{circuit_diagram} 
%shows the circuit proposed by the teacher to measure 
%voltage and current safely. 

\begin{figure}[h]
\centering
\includegraphics[clip,width=0.75\columnwidth]{circuit_diagram.png}
\caption{Circuito utilizado para la obtención \\
de los valores de potencia y tensión.}
\label{circuit_diagram}
\end{figure}

\textit{CH1} y
\textit{CH2} representan los canales del osciloscopio,
\textit{Phase} corresponde a la fase y \textit{N} corresponde
al neutro. \textit{V\textsubscript{ac}} representa 
la onda de 120 
V\textsubscript{rms} que entrega la red eléctrica. 
\textit{Fuse} representa un fusible cuyo valor se explicará 
más adelante, \textit{Rshunt} representa una resistencia 
de potencia de 1 $\Omega$ y 10 W. \textit{LOAD} corresponde
al dispositivo al que vamos al que se va a realizar la
medición.\\ 

Se puede observar del
gráfico que el valor de tensión que cae en el dispositivo
es la diferencia de los datos que se obtienen en los canales 
del osciloscopio. $V_D = CH1 - CH2$. Y la corriente que 
pasa a través del dispositivo es la razón entre la tensión 
que se obtiene en el canal \textit{CH2} y el valor de la 
resistencia. Esto es el valor del canal, puesto que 
la resistencia es de 1 $\Omega$. Esta información es muy
últil para los cálculos de potencia que se explicarán más 
adelante.\\

%obtained from the university phase line, \textit{Fuse}
%represents a 3A circuit breaker, \textit{Load} represents 
%the load and \textit{Rshunt} 
%represents the 1 $\Omega$ and 10 W 
%power resistor. We chose a small value for the resistor 
%to do not affect the functioning of the circuit. \\

%The gauge of the wire was selected depending on maximum 
%current needed by the higher power load according to 
%electrical parameters 
%of each device. Current values of devices are 
%shown on the Table \ref{current_table}. Based on this, 
%AWG 14 wire was selected to build the power meter, because 
%it is able to conduct a maximum of 15 A\textsubscript{rms}.
%In addition to this, the circuit uses a 3A fuse to protect
%load and wires of high currents.\\

Con respecto al calibre del alambre, fue seleccionado 
dependiendo de la máxima corriente que necesita cada 
dispoitivo, de acuerdo con los parámetros eléctricos 
que se muestran en la descripción del dispositivo. Estos
valores de corriente son mostrados en la Tabla
\ref{current_table}. \\

\begin{table}
\centering
\caption{Power and current values of each device.}
\begin{tabular}{|l|p{1.2cm}|p{1.5cm}|p{1.5cm}|}
\hline 
Dispositivo & Potencia (W) & Tensión (V\textsubscript{rms}) 
& Corriente (A\textsubscript{rms}) \\ \hline 
Cautil 	&  40	& 120 	& 0.33 \\ \hline 
Taladro 		& 130	& 120   & 1.08 \\ \hline 
Portátil 		& 140 	& 120   & 1.16 \\ \hline 
Bombillo		& 70  	& 120   & 0.58 \\ \hline 
Resistencia & 10 & - & 3.17 \\ \hline
\end{tabular}
\label{current_table}
\end{table}

The circuit is inside a 4x4 box with a fuse holder to 
change the breaker, 4 measuring terminals; the white 
one for neutral, green one for ground and 
both black one for phase. The box is shown in the 
Figure \ref{circuit_box}. \\

\begin{figure}[h]
\centering
\includegraphics[clip,width=\columnwidth]{circuit_box.png}
\caption{Circuit box.}
\label{circuit_box}
\end{figure}

For this practice three different loads are chosen to 
analize their voltage and current waveform: 
a soldering iron 
like resistive load, a drill like inductive load and 
a laptop like non-linear load. These loads are conected 
to an electrical outlet, using the power meter. Owing to 
high electrical network inertia, the voltage waveform do
not depend on the load. For this reason, the voltage signal
is similar in all measurings. However, current waveform 
depend on impedance load and voltage, and the first 
parameter is realted to load features. Therefore, the 
power analysis will be centered on current waveform.\\

\subsubsection{Solderin Iron (resistive load)}
The current waveform is a pure sine signal and it does not 
present any distortion and between voltage and current there
are not phase difference. This can be observed in the Figure
\ref{original_resistive_load}. 

\begin{figure}[h]
\centering
\includegraphics[clip,width=\columnwidth]{original_waveform_cautin.png}
\caption{Voltage and current waveforms of a resistive load.}
\label{original_resistive_load}
\end{figure}

Additionally, in a resistive load, product between 
current and voltage always 
is positive, this means that soldering iron is a power 
consumer and it is not able to supply energy. 

\begin{figure}[h]
\centering
\includegraphics[clip,width=\columnwidth]
{zoomed_current_furier_coefficients_resistive.png}
\caption{Fourier current coefficients for resistive load.}
\label{fourier_corrent_coefficients_resistive}
\end{figure}

Fourier Transform allow to identify harmonic components of
current signal. According to Figure
\ref{fourier_corrent_coefficients_resistive}, current 
waveform of a resistive load only have a 60 Hz component
and it does not have any different harmonic component. So, 
there are not distortion, the power factor \textit{PF} is 
high and Total Harmonic Distortion \textit{THD} is 
very small.
This is an important feature from resistive loads.  

\begin{lstlisting}[caption = Output for resistive load.]
T     =     0.0167 s 
f0    =    59.9520 Hz 
Vrms  =   123.3322 V
Irms  =     0.7860 A
S     =    96.9336 VA
Pavg  =    96.8309 W 
P     =    96.8309 W 
Q     =    -1.8282 VAR 
D_fast=     4.0701 VA 
D     =     4.0562 VA 
PF    =     0.9989 
THD_V =     1.8878 %
THD_I =     4.0154 %
\end{lstlisting}

\subsubsection{Drill (inductive load)} The current 
behaviour for inductive loads is different to resistive 
loads. Current waveform is not a pure sine signal, 
because there are harmonic components that change shape 
current signal. This can be observed in the Figure 
\ref{original_inductive_load} and 
\ref{fourier_corrent_coefficients_inductive}. 
The waveform present peaks as a result of 
current harmonic component: $3 f_0$, $5 f_0$, $7 f_0$ 
and $9 f_0$,  where $f_0$ is fundamental frquency (60 Hz). 
In addition, there are an 
phase difference betweent both signal: current signal is
slow respect to voltage signal.

\begin{figure}[h]
\centering
\includegraphics[clip,width=\columnwidth]
{original_waveform_drill.png}
\caption{Voltage and current waveforms of a inductive load.}
\label{original_inductive_load}
\end{figure}

In contrast to resistive load, the power factor value is 
very low and reactive power $Q$ and $THD_I$ are very high.
According to Listing \ref{output_inductive},
for this device, Reactive Power $Q$ is higher that Active 
Power $P$. 
This high $Q$ value mean that drill is a reactive element. 

\begin{figure}[h]
\centering
\includegraphics[clip,width=\columnwidth]
{zoomed_current_furier_coefficients_drill.png}
\caption{Fourier current coefficients for inductive load.}
\label{fourier_corrent_coefficients_inductive}
\end{figure}

\begin{lstlisting} [caption = {Output for inductive load.}
\label{output_inductive}]
T     =     0.0167 s 
f0    =    59.9520 Hz 
Vrms  =   123.8850 V
Irms  =     0.4831 A
S     =    59.8537 VA
Pavg  =    30.4372 W 
P     =    30.4372 W 
Q     =    38.7767 VAR 
D_fast=    33.9472 VA 
D     =    33.9434 VA 
PF    =     0.5085 
THD_V =     1.8561 %
THD_I =    69.7550 %
\end{lstlisting}

\subsubsection{Laptop (non-linear load)} Finally, 
according to Figure \ref{original_no_lineal_load},  
current signal of non-linear load do not present any 
sine shape. The current waveform present peaks more 
narrow that inductive peaks loads. Those peaks 
presents a high harmonic components which can be observed 
in the Figure \ref{fourier_corrent_coefficients_nonlinear}. 
The harmonic components of non-linear load are much higher 
than harmonic components of inductive load. 

\begin{figure}[h]
\centering
\includegraphics[clip,width=\columnwidth]
{original_waveform_computer.png}
\caption{Voltage and current waveforms of a non-linear load.}
\label{original_no_lineal_load}
\end{figure}

\begin{figure}[h]
\centering
\includegraphics[clip,width=\columnwidth]
{zoomed_current_furier_coefficients_computer.png}
\caption{Fourier current coefficients for non-linear load.}
\label{fourier_corrent_coefficients_nonlinear}
\end{figure}

Nevertheless, a laptop does not have reactives elements 
like elctrical coils, the power factor values is very 
low. The electronics circuits inside a laptop (or other 
complex electronic device) do not respond linear way. 
For this reason, current waveform presents a lot of 
distortion, as a result, Harmonic Power $D$ and $THD_I$ 
value are very high while $PF$ is small. 

\begin{lstlisting} [caption = Output for non-linear load.]
T     =     0.0167 s 
f0    =    59.9520 Hz 
Vrms  =   124.3447 V
Irms  =     0.3916 A
S     =    48.6915 VA
Pavg  =    23.9965 W 
P     =    23.9965 W 
Q     =    -5.7979 VAR 
D_fast=    41.9692 VA 
D     =    41.9644 VA 
PF    =     0.4928 
THD_V =     2.2668 %
THD_I =   164.9239 %
\end{lstlisting}

\subsection{AC contoller}

For the second part of the practice, a 
circuit using thyristors  was desinged to control 
the amount of power delivered to the light bulb. 
The input is the 120 V AC and 
the output is an AC waveform where there is a delay angle 
before triggering the AC voltage. The phase control is 
done with a potentiometer. The circuit designed is shown 
in the Figure \ref{ACcontroller}.

\begin{figure}[h]
\centering
\includegraphics[clip,width=\columnwidth]{controller.png}
\caption{AC controller circuit.}
\label{ACcontroller}
\end{figure}

\begin{table}
\caption{Power measures from diferents controller values}
\begin{tabular}{|p{1.2cm}|c|c|c|c|c|c|}
\hline 
 & P  & Q  & D  & S  & PF & THDI  \\
 &(W) & (VAR) &(VA) & (VA) &  & (\%) \\ \hline  
NoCtrl & 77.24 & -3 & 3 & 77.36 & 0.999 & 4.01 \\ 
\hline 
Pos 0 & 74.2 & 0.1 & 5.1 & 74.37 & 0.998 & 6.23 \\ 
\hline 
Pos 1 & 69.8 & 10.5 & 17.29 & 72.68 & 0.96 & 23.99 \\
\hline 
Pos 2 & 53.64 & 23.48 & 31.41 & 66.45 & 0.807 & 53.48 \\
\hline 
Pos 3 & 40.19 & 27.81 & 35.69 & 60.51 & 0.664 & 73.58 \\ 
\hline
Pos 4 & 30.47 & 28.41 & 37.36 & 55.96 & 0.545 & 89.63 \\ 
\hline 
\end{tabular}
\end{table}

\section{CONCLUSIONES}

%The resistive load shows a power factor of almost 1. 
%The inductive load has a curly current shape, thanks to 
%the type of motor being used in the drill. The 
%non-linear load shows a kinky wave shape that 
%depends on the circuit inside the electronic device.\\

%Power computations let understand better the 
%circuits, taking into account the meaning behind the 
%power factor and the different kinds of 
%power related to a circuit.\\

%The dimmer allows to regulate the power taken by the 
%light bulb. Taking into account that a light bulb is 
%resistive, the effectiveness depends on the efficiency 
%voltage and effective current. \\

%The circuit is controlling the AC power to the light bulb 
%by switching on and off during the positive and negative 
%regions of the input sinusoidal signal. During the 
%negative part of the input signal, the same type of 
%response will be obtained since both the DIAC and TRIAC 
%can be triggered in the reverse direction. Varying the 
%resistance R, it is possible to control the driving 
%angle. [3]\\

%About inductive and non-linear loads, you shouldn't use 
%the dimmer circuit, because both rely on the PF and the
%voltage and current wave shape.\\

%Electronic devices, such as,
%DIACs and TRIACs let us build circuits to control the 
%power taken by a resistive load.\\

La carga resistiva muestra un factor de potencia de 
casi unitario. La carga inductiva tiene una forma de 
corriente rizada gracias al tipo de motor que se 
utiliza en el taladro. La carga no lineal muestra 
una forma de onda rizada que depende del circuito 
dentro del dispositivo electrónico. \\

El dimmer permite regular la potencia de la bombilla. 
Teniendo en cuenta que una bombilla es resistiva, la 
efectividad depende del voltaje y la corriente que se 
encuentra en éste. \\

El circuito controla la alimentación de la bombilla 
encendiendo y apagando durante las regiones positiva y 
negativa de la señal sinusoidal de entrada. Durante 
la parte negativa de la señal de entrada, se obtendrá 
el mismo tipo de respuesta, ya que tanto el DIAC como 
el QUADRAC se pueden disparar en la dirección inversa. 
Al variar la resistencia R, es posible controlar el
ángulo de conducción.\\

Acerca de las cargas inductivas y no lineales, no se 
debe utilizar el circuito dimmer, ya que ambos dependen 
del factor de potencia y de la forma de onda de tensión 
y corriente. \\

Los dispositivos electrónicos, como el DIAC y el QUADRAC, 
permiten construir circuitos para controlar la potencia 
de una carga resistiva.

\begin{thebibliography}{1}

\bibitem{IEEEhowto:kopka}
H.~Kopka and P.~W. Daly, \emph{A Guide to {\LaTeX}}, 3rd~ed.\hskip 1em plus
  0.5em minus 0.4em\relax Harlow, England: Addison-Wesley, 1999.

\end{thebibliography}

\end{document}