\documentclass[journal]{IEEEtran}
\usepackage[spanish]{babel}
\usepackage[utf8]{inputenc}
\usepackage{graphicx}
\graphicspath{ {images/} }
%\usepackage{enumerate}
\usepackage{amsmath}
\usepackage{mathtools}
\usepackage{float}
\usepackage{amssymb}
\usepackage{listings}
\usepackage{hyperref}
%\usepackage{caption}
\usepackage[justification=centering]{caption}
\usepackage[font=scriptsize,labelfont=bf]{caption}
\renewcommand{\IEEEkeywordsname}{Palabras clave}
%\renewcommand{\table}{TABLA}
\renewcommand\spanishtablename{Tabla}
\renewcommand\lstlistingname{Salida}

\bgroup
\def\arraystretch{1.5}

\hypersetup{
    colorlinks=true,
    linkcolor=black,
    filecolor=magenta,      
    urlcolor=blue,
}


\lstset{
basicstyle=\footnotesize
}

\title{\textbf{Potencia en aplicaciones domésticas}}
\author{
  \IEEEauthorblockN{ Jorge Lambra\~no$^3$, Julian Rojas$^2$, 
  Juan Sánchez$^3$}\\
  \IEEEauthorblockA{\texttt{\small{$^1$jelambrano, $^2$drojasj, $^3$paradac @uninorte.edu.co
   }}}}

%Encabezado de reporte: Edite con su información.
%\asignatura      {Mediciones e Instrumentación IEN 8435}
%\grupoLab        {1}
%\autores         {Kevin Acuña, María De La Rosa, 
%                  Luis Torres, Gloria Zúñiga}
%\uninorteEmails  {kacuna, mdlrosa, ltorres, gzuniga}
%\title          {Lab 1: Título de la primera práctica}

\begin{document}

\maketitle

\begin{abstract}  
En el presente informe se exponen los resultados de las
mediciones de potencia, tensión y corriente de diferentes
dispositivos, asimismo se encuentra el diseño de un 
circuito que permite controlar la potencia que disipa una 
carga resistiva, utilizando DIACs y TRIACs. 
\end{abstract}

%-----------------------------------------------------------

\begin{IEEEkeywords}  
%Current, Power, Power Factor, Voltage, Waveform.
Armónico, Corriente, Distorsión, Factor de Potencia,  Potencia, Tensión, Transformada de Fourier, Voltaje. 
\end{IEEEkeywords}

\section{INTRODUCCIÓN}

El objetivo principal de esta práctica es realizar 
un análisis de la forma de onda y las mediciones 
de tres tipos diferentes de carga.
Todos los dispositivos electrónicos están compuestos 
de resistencias, condensadores e inductancias, un 
soldador es una carga lineal resistiva, requieren calor 
para funcionar. El voltaje medido sería el mismo que 
el de la fuente, pero la corriente variará según el 
consumo de energía del dispositivo. Se espera la misma 
forma de onda para el voltaje y la corriente y ningún 
cambio de fase entre ellos.\\

Un taladro es una carga lineal inductiva, basado 
en el hecho de que los motores están hechos de 
bobinas. Debería haber un cambio de fase entre el 
voltaje y la corriente. La computadora portátil 
es una carga no lineal, la forma de onda del voltaje 
sería la misma, pero esperamos una forma diferente 
para la corriente. $^{\left[1\right]}$ \\

La segunda parte consiste en diseñar y desarrollar 
un controlador de CA compuesto por un DIAC y un 
QUADRAC. Este tipo de circuito puede cambiar el 
voltaje RMS en los terminales de una carga lineal 
manipulando el ángulo de disparo del QUADRAC usando 
un potenciómetro. La carga ya no será lineal debido 
al circuito resultante entre la bombilla y el 
controlador AC.


%--------------------------------------------------------------------

\section{PROCEDIMIENTO Y ANÁLISIS DE RESULTADOS}

\subsection{Cálculos de Potencia}

El objetivo de esta práctica es medir el consumo eléctrico 
de dispositivos cotidianos, tales como computadores 
portátiles, un pequeño taladro y un cautin. Teniendo en 
cuenta los valores altos de tensión y corriente que se 
pueden encontrar en instalaciones eléctricas domésticas, se
tomaron todas las precausiones necesarias para garantizar 
seguridad a los dispositivos, a los instrumentos de 
medición y a las personas que realizan la medición. 
Para poder obtener los valores de tensión, corriente y 
corriente se 
diseñó un circuito que junto con el osciloscopio, 
permite la obtención de valores de tensión y corriente. El 
diseño del circuito se encuentra ilustrado en la Figura 
\ref{circuit_diagram}. 

\begin{figure}[h]
\centering
\includegraphics[clip,width=0.7\columnwidth]{circuit_diagram.png}
\caption{Circuito utilizado para la obtención \\
de los valores de corriente y tensión.}
\label{circuit_diagram}
\end{figure}

\textit{CH1} y \textit{CH2} representan los puntos donde
el osciloscopio toma las medidas a través de las 
sondas de los canales 1 y 2,
\textit{Phase} y \textit{N} corresponden a la fase y 
al neutro de la red a la que se conecta el circuito. 
\textit{V\textsubscript{ac}} representa 
la onda de 120V\textsubscript{rms} 
que entrega la red eléctrica. 
\textit{Fuse} representa un fusible cuyo valor se explicará 
más adelante, \textit{Rshunt} representa una resistencia 
de potencia de 1 $\Omega$ y 10 W. \textit{LOAD} corresponde
al dispositivo al que se va a realizar la
medición.\\ 

Se puede observar del
gráfico que el valor de tensión que cae en el dispositivo
es la diferencia de los datos que se obtienen en los canales 
del osciloscopio. $V_L = CH1 - CH2$. Y también,  
la corriente que 
pasa a través del dispositivo es la razón entre la tensión 
que se obtiene en el canal \textit{CH2} y el valor de la 
resistencia. Esto es el valor del canal, puesto que 
la resistencia es de 1 $\Omega$. Esta información es muy
últil para los cálculos de potencia que se explicarán más 
adelante.

\begin{table}[h]
\centering
\begin{tabular}{|l|p{1.2cm}|p{1.5cm}|p{1.5cm}|}
\hline 
Dispositivo & Potencia (W) & Tensión (V\textsubscript{rms}) 
& Corriente (A\textsubscript{rms}) \\ \hline 
Cautil 	&  40	& 120 	& 0.33 \\ \hline 
Taladro 		& 130	& 120   & 1.08 \\ \hline 
Portátil 		& 140 	& 120   & 1.16 \\ \hline 
Bombillo		& 70  	& 120   & 0.58 \\ \hline 
Resistencia & 10 & - & 3.17 \\ \hline
\end{tabular}
\caption{Valores de potencia y corriente para cada dispositivo.}
\label{current_table}
\end{table}

Con respecto al calibre del alambre, fue seleccionado 
dependiendo de la máxima corriente que necesita cada 
dispoitivo de acuerdo con los parámetros eléctricos 
que se muestran en la descripción del dispositivo. Estos
valores de corriente son mostrados en la Tabla
\ref{current_table}. Basándose en la 
Tabla \ref{current_table} se seleccionó un alambre 
AWG 14 capaz de soportar una corriente máxma de 
15A\textsubscript{rms} puesto que es que utiliza 
normalmente en instalaciones eléctricas domésticas. 
La corriente máxima que sorporta el fusible es 
3A\textsubscript{rms}, se escogió este valor basándose
en la máxima corriente que es capaz de soportar la 
resistencia. \\

El circuito se introdujo dentro de una caja 4x4, el 
fusible se ubicó de tal forma que no fuese necesario 
abrir la caja para reemplazarlo. 
Además se colocaron 
cuatro borneras para poder medir de forma más segura. 

\begin{figure}[h]
\centering
\includegraphics[clip,width=0.7\columnwidth]{circuit_box.png}
\caption{Imagen del montaje del circuito.}
\label{circuit_box}
\end{figure}

En esta práctica se utilizaron tres cargas diferentes 
para analizar las formas de onda de corriente y tensión: 
Un \textit{cautin} como ejemplo de una carga puramente 
resistiva, un \textit{taladro} como carga inductiva y un 
\textit{computador portátil} como carga no lineal. 
Gracias a la inercia de la red eléctrica, la onda de 
tensión no depende de la carga, esto hace que la medida de 
la tensión sea similar para todas las cargas. Sin embargo, 
la onda de corriente depende de la impedancia de la carga. 
Por esta razón, el analisis de potencia se centrará 
en la forma de onda de la corriente, y se espera que 
la distorsión en la onda de tensión sea mínima. En otras 
palabras, mientras el valor de la Distorsión Total Armónica
de Tensión \textit{THD\textsubscript{V}} se mantenga más o
menos constante, el valor de la Distorsión Total Armónica
de Corriente \textit{THD\textsubscript{I}} varíe segun 
la carga que se esté midiendo.  \\

\begin{figure}[h]
\centering
\includegraphics[clip,width=0.9\columnwidth]{original_waveform_cautin.png}
\caption{Ondas de tensión y corriente para una carga
resistiva.}
\label{original_resistive_load}
\end{figure}

\subsubsection{Cautin (carga resistiva)}
La onda de corriente para esta carga fue una señal 
sinusoidal que no presentaba ninguna distorsión y además 
se encontraba en fase con la onda de voltaje. Las imágenes
de las formas de onda que se obtivieron en el osciloscopio 
se observan en la Figura \ref{original_resistive_load}. 
Adicionalmente, en una carga resistiva, el producto de la 
onda de corriente y tensión siempre es positivo, esto 
significa que un cautin funciona siempre como un 
consumidor de potencia. \\

\begin{figure}[h]
\centering
\includegraphics[clip,width=0.9\columnwidth]
{zoomed_current_furier_coefficients_resistive.png}
\caption{Coeficientes de Fourier de la onda de corriente \\
para una carga resistiva.}
\label{fourier_corrent_coefficients_resistive}
\end{figure}

De acuerdo con la Transformada de Fourier de la Figura 
\ref{fourier_corrent_coefficients_resistive}, la onda 
de corriente solamente tiene la componente de la frecuencia 
fundamental (60Hz), las demás son despresiables. 
Por tanto, no hay distorsión, el factor de potencia 
\textit{PF} es casi unitario, mientras que la Distorsión 
Total Armónica de Corriente 
\textit{THD\textsubscript{I}} es muy pequeña. Esto lo 
confirma la Salida 1, en donde el Factor de Potencia es 
0.999, la potencia activa \textit{P} es muy alta en 
comparación a la potencia reactiva \textit{Q}, la 
potencia de distorsión \textit{D} 
y la Distorsión Armónica de Corriente 
\textit{THD\textsubscript{I}} es 4\%. 


\begin{lstlisting}[caption = Carga resistiva.]
T     =     0.0167 s 
f0    =    59.9520 Hz 
Vrms  =   123.3322 V
Irms  =     0.7860 A
S     =    96.9336 VA
Pavg  =    96.8309 W 
P     =    96.8309 W 
Q     =    -1.8282 VAR 
D_fast=     4.0701 VA 
D     =     4.0562 VA 
PF    =     0.9989 
THD_V =     1.8878 %
THD_I =     4.0154 %
\end{lstlisting}

\subsubsection{Taladro (carga inductiva)}

\begin{figure}[h]
\centering
\includegraphics[clip,width=0.9\columnwidth]
{original_waveform_drill.png}
\caption{Ondas de tensión y corriente para una carga
inductiva.}
\label{original_inductive_load}
\end{figure}

La forma de la corriente no es una puramente sinusoidal,
aparecen compoenentes armónicas que distorsionan la forma 
de la señal. La forma de onda de la corriente
se puede observar en la Figura 
\ref{original_inductive_load}.

\begin{figure}[h]
\centering
\includegraphics[clip,width=0.9\columnwidth]
{zoomed_current_furier_coefficients_drill.png}
\caption{Coeficientes de Fourier de la  
onda de corriente \\ para una carga inductiva.}
\label{fourier_corrent_coefficients_inductive}
\end{figure}

La presencia de las componentes armónicas se observa 
con mayor claridad gracias a la Transformada de Fourier
en la Figura 
\ref{fourier_corrent_coefficients_inductive}.
Según ésta, la señal de corriente 
presenta fuertes componentes armónicas en las frecuencias
$3 f_0$, $5 f_0$, $7 f_0$  y $9 f_0$, donde $f_0$ es la 
frecuencia fundamental (60Hz). Además de las componentes 
armónicas existe un desfase entre la onda de corriente 
y tensión. Este desfase es una característica importante 
de las cargas inductivas. \\

%\subsubsection{Drill (inductive load)} The current 
%behaviour for inductive loads is different to resistive 
%loads. Current waveform is not a pure sine signal, 
%because there are harmonic components that change shape 
%current signal. This can be observed in the Figure 
%\ref{original_inductive_load} and 
%\ref{fourier_corrent_coefficients_inductive}. 
%The waveform present peaks as a result of 
%current harmonic component: $3 f_0$, $5 f_0$, $7 f_0$ 
%and $9 f_0$,  where $f_0$ is fundamental frquency (60 Hz). 
%In addition, there are an 
%phase difference betweent both signal: current signal is
%slow respect to voltage signal.

A diferencia de una carga resistiva, la mediciones y 
los cálculos de potencia en un taladro dieron como 
resultado un valor mucho más alto en la  
Potencia Reactiva $Q$ y en la Distorsión Total Armónica 
de Corriente. Tanto así que hacen 
que el valor del Factor de Potencia $pf$ sea muy bajo, 
aproximadamente 0.51. Los motores y otras cargas 
inductivas, por lo general tienen un factor de potencia 
más alto (entre 0.75 y 0.9), una causa por la cual el 
factor de potencia durante la medición fue mucho más 
bajo que lo esperado es la presencia de un circuito 
electrónico que regula la velocidad y el torque del 
taladro. Este circuito electrónico genera distorsión en la 
señal de corriente la cual tiene el efecto de  
aumentar el \textit{THD\textsubscript{I}} y la Potencia 
de Distorsión $D$ y por tanto, reducir el Factor de 
Potencia. En la segunda parte de este informe se analizará
más detallamente el efecto de circuitos eléctronicos 
controladores de potencia. 

\begin{lstlisting} [caption = {Carga inductiva.}
\label{output_inductive}]
T     =     0.0167 s 
f0    =    59.9520 Hz 
Vrms  =   123.8850 V
Irms  =     0.4831 A
S     =    59.8537 VA
Pavg  =    30.4372 W 
P     =    30.4372 W 
Q     =    38.7767 VAR 
D_fast=    33.9472 VA 
D     =    33.9434 VA 
PF    =     0.5085 
THD_V =     1.8561 %
THD_I =    69.7550 %
\end{lstlisting}

\subsubsection{Computador portátil (carga no-lineal)} 
Finalmente,  la forma de onda de la corriente y de la 
tensión se ilustran en la Figura 
\ref{original_no_lineal_load}. Se observa, en primer lugar
la presencia de picos bastante estrechos en la onda  
de corriente, esta forma de consumir 
es típica de muchos dispositivos electrónicos. 
La distorsión de la onda de corriente es tan alta que ya 
no es posible identificar una forma sinusoidal. 
Lo estrecho de los picos da una idea de la presencia de 
componentes armónicas con valores bastante altos. 

%according to Figure \ref{original_no_lineal_load},  
%current signal of non-linear load do not present any 
%sine shape. The current waveform present peaks more 
%narrow that inductive peaks loads. Those peaks 
%presents a high harmonic components which can be observed 
%in the Figure \ref{fourier_corrent_coefficients_nonlinear}. 
%The harmonic components of non-linear load are much higher 
%than harmonic components of inductive load. 

\begin{figure}[h]
\centering
\includegraphics[clip,width=0.9\columnwidth]
{original_waveform_computer.png}
\caption{Ondas de tensión y corriente para una carga
no-lineal.}
\label{original_no_lineal_load}
\end{figure}

Si se aplica la Trasformada de Fourier a la señal de 
corriente, se obtiene la Figura 
\ref{fourier_corrent_coefficients_nonlinear}. Se observa
que la magnitud de la segunda componente de frecuencia 
es mayor que la componente de la frecuencia fundamental. 
Y las compoenentes armónicas siguientes 
también tienen un valor bastante alto. \\

\begin{figure}[h]
\centering
\includegraphics[clip,width=0.9\columnwidth]
{zoomed_current_furier_coefficients_computer.png}
\caption{Coeficientes de Fourier de la onda de corriente\\
para una carga no-lineal.}
\label{fourier_corrent_coefficients_nonlinear}
\end{figure}

A pesar que el computador portátil no posee grandes 
elementos reactivos, en comparación al motor de un taladro, 
el Factor de Potencia que se obtuvo fue muy bajo. 
Esto se debe a que los circuitos electrónicos no presentan
un comportamiento lineal en el consumo de la corriente, lo 
cual  
es la causa de la alta distorisión de la onda de corriente. 
Como resultado, el valor del Factor de Potencia \textit{pf}
es bajo, mientras que los valores de la 
Potencia de Distorsión $D$ y de
la Distorsión Total Armónica de Corriente son muy altos. 
En la Salida 3, pueden observarse todos estos valores. 

\begin{lstlisting} [caption = carga no-lineal.]
T     =     0.0167 s 
f0    =    59.9520 Hz 
Vrms  =   124.3447 V
Irms  =     0.3916 A
S     =    48.6915 VA
Pavg  =    23.9965 W 
P     =    23.9965 W 
Q     =    -5.7979 VAR 
D_fast=    41.9692 VA 
D     =    41.9644 VA 
PF    =     0.4928 
THD_V =     2.2668 %
THD_I =   164.9239 %
\end{lstlisting}

\subsection{Controlador AC}

\begin{figure}[h]
\centering
\includegraphics[clip,width=0.6\columnwidth]{controller.png}
\caption{Circuito del controlador AC.}
\label{ACcontroller}
\end{figure}

Para la segunda parte de esta práctica, se diseñó un 
circuito usando tiristores para controlar por medio 
de un potenciómetro la cantidad 
de potencia consumida por un bombillo. 
El circuito que se utilizó se muestra 
en la Figura \ref{ACcontroller}. 
Cuando el voltaje en el capacitor alcanza el 
voltaje de ruptura del DIAC, éste se encuentra 
parcialmente descargado por el DIAC a través de la 
compuerta del QUADRAC. En ese momento el QUADRAC 
entra en modo conducción en el ciclo de media onda. 
En este circuito el QUADRAC se encuentra en los 
cuadrantes I y III. La simplicidad de este circuito 
hace que sea accesible para aplicaciones con pequeño 
control de rango. Los circuitos dimmer se basan en el 
control de potencia que se logra variando el ángulo de 
conducción del QUADRAC, de 30$^{\circ}$ a 160$^{\circ}$ 
cuando fluye la 
corriente por un potenciómetro que tiene la función de 
controlarla. Al variar el potenciómetro, es 
posible controlar el
ángulo de conducción.$^{\left[2\right]}$ 

\begin{figure}[h]
\centering
\includegraphics[clip,width=0.9\columnwidth]{controller_values.png}
\caption{Señales de corriente para diferentes valores \\
del controlador.}
\label{ACcontroller_plots}
\end{figure}

Se realizaron las mismas mediciones con el controlador AC 
que se utilizaron anteriormente. se obtuvieron variaciones 
según la posición del potenciómetro, de tal forma que en 
la medida que el valor del potenciómetro se hacía más
pequeño, la onda se distorsionaba más. Esto se observa 
claramente en la Figura \ref{ACcontroller_plots}. 
Además, se compararon los valores de
Factor de Potencia, Potencia Activa, Potencia de Distorsión
y Distorsión Armónica, para el caso donde no hay controlador, 
y para diferentes valores de resistencia (posiciones del 
potenciómetro). Los resultados 
de esas comparaciones se observan en la Tabla 
\ref{table:controller}. 


\begin{table}[h]
\begin{tabular}{|p{1.2cm}|c|c|c|c|c|c|}
\hline 
 & P  & Q  & D  & S  & PF & THDI  \\
 &(W) & (VAR) &(VA) & (VA) &  & (\%) \\ \hline  
NoCtrl & 77.24 & -3 & 3 & 77.36 & 0.999 & 4.01 \\ 
\hline 
Pos 0 & 74.2 & 0.1 & 5.1 & 74.37 & 0.998 & 6.23 \\ 
\hline 
Pos 1 & 69.8 & 10.5 & 17.29 & 72.68 & 0.96 & 23.99 \\
\hline 
Pos 2 & 53.64 & 23.48 & 31.41 & 66.45 & 0.807 & 53.48 \\
\hline 
Pos 3 & 40.19 & 27.81 & 35.69 & 60.51 & 0.664 & 73.58 \\ 
\hline
Pos 4 & 30.47 & 28.41 & 37.36 & 55.96 & 0.545 & 89.63 \\ 
\hline 
\end{tabular}
\caption{Mediciones de potencia para diferentes \\ valores 
del Potenciómetro.}
\label{table:controller}
\end{table}

Básandose en la información se pudo observar que el 
controlador permite variar la potencia activa que 
consume el dispositivo, es decir, reduce o aumente 
el brillo de la lámpara. Sin embargo a medida que se 
reduce el brillo del bombillo aumenta la Potencia de 
Distorsión $D$, la Distorsión en la Corriente y se 
reduce el Factor de Potencia. Esto sucede porque a
medida que se aumenta el retraso del disparo, la onda 
deja de ser sinusoidal y aparecen componentes armónicas 
que la distosiones, epecialmente, las que se 
generan por el pico del disparo. 

\section{CONCLUSIONES}

De acuerdo con los resultados obtenidos en esta práctica, 
se puede concluir que las cargas resistivas muestran 
un factor de potencia de 
casi unitario. En cambio, una carga inductiva, como un 
taladro, inductiva tiene una forma de 
corriente rizada gracias al tipo de motor que se 
utiliza en el taladro. Y también, una carga no lineal 
muestra una forma de onda rizada que depende del circuito 
dentro del dispositivo electrónico. \\

El dimmer permite regular la potencia de la bombilla. 
Teniendo en cuenta que una bombilla es resistiva, la 
efectividad depende del voltaje y la corriente que se 
encuentra en éste. El circuito controla la alimentación 
de la bombilla 
encendiendo y apagando durante las regiones positiva y 
negativa de la señal sinusoidal de entrada. Durante 
la parte negativa de la señal de entrada, se obtendrá 
el mismo tipo de respuesta, ya que tanto el DIAC como 
el QUADRAC se pueden disparar en la dirección inversa. \\

Los dispositivos electrónicos, como el DIAC y el QUADRAC, 
permiten construir circuitos para controlar la potencia 
de una carga resistiva.
Acerca de las cargas inductivas y no lineales, no se 
debe utilizar el circuito dimmer, ya que ambos dependen 
del factor de potencia y de la forma de onda de tensión 
y corriente. 

\begin{thebibliography}{1}

\bibitem{IEEEhowto:Rochelle}
S. Rochelle and T. Brian, \textit{Triac light dimmer}, University of Texas, Austin, TX, Rep., Feb. 2, 2005. Available at: 
\url{http://us- ers.ece.utexas.edu/~kwasin-ski/_1_EE362L_Sample_Report_Light_Dim-mer.pdf }

\bibitem{IEEEhowto:hart}
H. W. Daniel, “Power Computations," in \textit{Power Electronics}, Valparaiso, IN: McGraw-Hill, 2011, pp. 21-54.

\end{thebibliography}

\end{document}


%[2] H. W. Daniel, “Power Computations,” in Power Electronics, Valparaiso, IN: McGraw-Hill, 2011, pp. 21-54.
%[3]https://es.wikipedia.org/wiki/Triac
